\begin{frame}
  \frametitle{(Section 1.1) Program Analysis}

  \begin{block}{Goal}
    Predict {\color{blue}safe} and {\color{blue}computable approximations} of the possible
    behaviours that arise at runtime when executing a program. \pause
    Usage Scenarios:
    \begin{itemize}
     \item program optimization
     \item program transformation
     \item program design metrics
     \item program security
    \end{itemize}
  \end{block}
\end{frame}

\begin{frame}
  \frametitle{(Section 1.2) The While Language}

  \begin{block}{Features}
    \begin{itemize}
     \item tiny (imperative) programming language
     \item integer and boolean expressions  
     \item conditionals and loops \pause
     \item every statement has a unique label  
    \end{itemize}
  \end{block}

  \begin{itemize}
    \item running example: reaching definitions. 
  \end{itemize}
\end{frame}

\begin{frame}
  \frametitle{(Section 1.4) An introduction to DataFlow Analysis}

  \begin{enumerate}
    \item The program is modeled as a (control-flow) graph: the nodes are the
  the elementary blocks (statements) and the edges describe how the
  control might pass from one statement to another.

    \item We define functions to every node, so that data flow information
      can be computed using pairs of \emph{Gen} and \emph{Kill} abstractions---considering the
      information produced in the previou(s) node(s). What should we do
      at merge nodes?
  \end{enumerate}

  \pause Computing a solution for a dataflow problem often
  requires multiple iteractions, where every new iteration
  provides a better approximation (monotone framework).
  The iteration continues until achieving a {\color{blue}fixpoint}.
  \pause $f(s) = s$
  
\end{frame}

\begin{frame}[fragile]
  \frametitle{Running Example}
  
  \begin{center}
  	\large Example 01: Factorial
  \end{center}
  
  \begin{columns}
    \column[t]{0.40\textwidth}
While language
    
\begin{verbatim}
y := x;           (1)
z := 1;           (2)
while(y > 1) do   (3)
  z := z * y;     (4) 
  y := y - 1;     (5) 
y := 0;           (6) 
\end{verbatim}
\column[t]{0.60\textwidth}
\pause Control Flow Graph
\digraph[scale=0.40]{cfg01}{
  node [fontname = "Handlee"];
  edge [fontname = "Handlee"];

  n1 [
    label = "y := x";
    shape = rect;
    xlabel="1";
  ];
  n2 [
    label = "z := 1";
    shape = rect;
    xlabel="2";
  ];
  n3 [
    label = "while y > 1";
    shape = diamond;
    xlabel="3";
  ];
  n4 [
    label = "z := z * y";
    shape = rect;
    xlabel="4";
  ];
  n5 [
    label = "y := y - 1";
    shape = rect;
    xlabel="5";
  ];
  n6 [
    label = "y := 0"; 
    shape = rect;
    xlabel="6";
  ]; 
  n1 -> n2;
  n2 -> n3;
  n3 -> n4[label = "yes"]
  n4 -> n5; 
  n5 -> n3; 
  n3 -> n6 [label = "no"];
  
  {
    rank=same;
    n3;n4;
  }
}

\end{columns}
\end{frame}