\begin{frame}
  \begin{huge}
    Available expressions
  \end{huge}
  \pause

  \vskip+1.5em
  
  \begin{itemize}
  \item Given a program point $u$, this algorithm identifies
    the expressions whose results at $u$ are the same as their
    previously computed values (regardless of the execution paths
    that reach $u$)~\cite{df-book}. 

  \item An expression \texttt{x + y} is available at a program point $u$
    if {\color{blue}every path} from the start node to $u$ evaluates \texttt{x + y},
    and after the last evaluation there is no subsequent assignment either to
    \texttt{x} or \texttt{y}~\cite{compilers-book}. 
  \end{itemize}
\end{frame}

\begin{frame}
  Questions:

  \begin{itemize}
   \item what is the ``shape'' of the {\color{blue}abstraction} data type ? \pause
   \item what {\color{blue}meet} operator is the best fit for our problem?
   \item what {\color{blue}CFG} should we use (forward or backward)?   
   \item how would you define the {\color{blue}Gen} and {\color{blue}Kill} functions?   
  \end{itemize}
  
\end{frame}


\begin{frame}
  \frametitle{Available Expressions}

  \begin{block}{Informal definition}
   For every vertice \texttt{(from, to)} in the control flow,
   we check if \texttt{assignment(v, exp) := to} holds (\texttt{to} is
   an assignment).\pause If this is the case, we remove all expressions that
   use the variable \texttt{v} ({\color{blue}Kill}) from the abstraction. For every
   statement, we include ({color{blue}Gen}) into the abstraction every (complex)
   expression that uses a variable. 
  \end{block} 
\end{frame}

\begin{frame}[fragile]
  \frametitle{Available Expressions} 

  \begin{block}{Equations}
    \begin{itemize}
    \item $Out(s) = Gen(s) \cup (In(s) - Kill(s))$  
    \item $In(s) = \bigcap_p Out(p), p \in pred(s), s \in stmts$
    \end{itemize}
  \end{block}

  \pause

\begin{block}{Iterative algorithm}
\begin{small}
\begin{verbatim}
// initialization
Out(Start) = {};
for(s in stmts) do Out(s) = U   // all complex expressions

while(changes to any out occur) do  
  for(s in stmts)
    // compute In(s) 
    // compute Out(s)   
\end{verbatim}
\end{small}
\end{block}
\end{frame}

\begin{frame}[fragile]
  \frametitle{Consider the example from~\cite{ppa-book}}

    \begin{columns}
    \column[t]{0.5\textwidth}
Example 02 (WHILE language)
    
\begin{verbatim}
x := a + b;           (1)
y := a * b;           (2)
while(y > a + b) do   (3)
  a := a + 1;         (4) 
  x := a + b;         (5) 
\end{verbatim}

\column[t]{0.5\textwidth}
\pause Control Flow Graph
\digraph [scale=0.40]{cfg02}{
node [fontname = "Handlee"];
edge [fontname = "Handlee"];

start [
 label = "start";
 shape = rect; 
];

n1 [
 label = "x := a + b";
 shape = rect;
 xlabel="1";
];

n2 [
 label = "y := a * b";
 shape = rect;
 xlabel="2";
];

n3 [
 label = "while y > a + b";
 shape = diamond;
 xlabel="3";
];

n4 [
 label = "a := a + 1";
 shape = rect;
 xlabel="4";
];

n5 [
 label = "x := a + b";
 shape = rect;
 xlabel="5";
];

end [
 label = "End";
 shape = rect; 
];

start -> n1;
n1 -> n2;
n2 -> n3;
n3 -> n4[label = "yes"]
n4 -> n5;
n5 -> n3;
n3 -> end [label = "no"];


{
rank=same;
n3;n4;
}
}
\end{columns}
  
\end{frame}



\begin{frame}[fragile, t]
 \frametitle{Iteration 1} 

\begin{center}
\begin{scriptsize}
\begin{minipage}{8cm}
    \begin{block}{Equations}
    \begin{itemize}
        \item $Out(s) = Gen(s) \cup (In(s) - Kill(s))$  
	    \item $In(s) = \bigcap_p Out(p), p \in pred(s), s \in stmts$
    \end{itemize}
    \end{block}
\end{minipage}
\end{scriptsize}
\end{center}

\begin{columns}[T]
\begin{column}[T]{.5\textwidth}
    \vspace{0pt}
    \digraph [scale=0.40]{cfg02}{
node [fontname = "Handlee"];
edge [fontname = "Handlee"];

start [
 label = "start";
 shape = rect; 
];

n1 [
 label = "x := a + b";
 shape = rect;
 xlabel="1";
];

n2 [
 label = "y := a * b";
 shape = rect;
 xlabel="2";
];

n3 [
 label = "while y > a + b";
 shape = diamond;
 xlabel="3";
];

n4 [
 label = "a := a + 1";
 shape = rect;
 xlabel="4";
];

n5 [
 label = "x := a + b";
 shape = rect;
 xlabel="5";
];

end [
 label = "End";
 shape = rect; 
];

start -> n1;
n1 -> n2;
n2 -> n3;
n3 -> n4[label = "yes"]
n4 -> n5;
n5 -> n3;
n3 -> end [label = "no"];


{
rank=same;
n3;n4;
}
}
    \end{column}
    \begin{column}[T]{.5\textwidth}
\vspace{30pt}    
	\begin{scriptsize}
	   \begin{table}[]
\begin{tabular}{|l|l|l|}
\hline
n & IN{[}n{]} & OUT{[}n{]} \\ \hline
1  &  \{ \}               & \{ a + b \}        \pause \\ \hline
2  &  \{ a + b \}         & \{ a + b, a * b \} \pause \\ \hline
3  &  \{ a + b, a * b \}  & \{ a + b, a * b \} \pause \\ \hline
4  &  \{ a + b, a * b \}  & \{ \}              \pause \\ \hline
5  &  \{ \}               & \{ a + b \}        \\ \hline
\end{tabular}
\end{table}   
	\end{scriptsize}
	\end{column}
%\hfill
    
\end{columns}

\end{frame}


\begin{frame}[fragile, t]
 \frametitle{Iteration 2 (fixed point)} 

\begin{center}
\begin{scriptsize}
\begin{minipage}{8cm}
    \begin{block}{Equations}
    \begin{itemize}
        \item $Out(s) = Gen(s) \cup (In(s) - Kill(s))$  
	    \item $In(s) = \bigcap_p Out(p), p \in pred(s), s \in stmts$
    \end{itemize}
    \end{block}
\end{minipage}
\end{scriptsize}
\end{center}

\begin{columns}[T]
\begin{column}[T]{.5\textwidth}
    \vspace{0pt}
    \digraph [scale=0.40]{cfg02}{
node [fontname = "Handlee"];
edge [fontname = "Handlee"];

start [
 label = "start";
 shape = rect; 
];

n1 [
 label = "x := a + b";
 shape = rect;
 xlabel="1";
];

n2 [
 label = "y := a * b";
 shape = rect;
 xlabel="2";
];

n3 [
 label = "while y > a + b";
 shape = diamond;
 xlabel="3";
];

n4 [
 label = "a := a + 1";
 shape = rect;
 xlabel="4";
];

n5 [
 label = "x := a + b";
 shape = rect;
 xlabel="5";
];

end [
 label = "End";
 shape = rect; 
];

start -> n1;
n1 -> n2;
n2 -> n3;
n3 -> n4[label = "yes"]
n4 -> n5;
n5 -> n3;
n3 -> end [label = "no"];


{
rank=same;
n3;n4;
}
}
    \end{column}
    \begin{column}[T]{.5\textwidth}
\vspace{30pt}    
	\begin{scriptsize}
	   \begin{table}[]
\begin{tabular}{|l|l|l|}
\hline
n & IN{[}n{]} & OUT{[}n{]} \\ \hline
1  &  \{ \}        & \{ a + b \}        \pause \\ \hline
2  &  \{ a + b \}  & \{ a + b, a * b \} \pause \\ \hline
3  &  \{ a + b \}  & \{ a + b \}        \pause \\ \hline
4  &  \{ a + b \}  & \{ \}              \pause \\ \hline
5  &  \{ \}        & \{ a + b \}        \\ \hline
\end{tabular}
\end{table}   
	\end{scriptsize}
	\end{column}
%\hfill
    
\end{columns}

\end{frame}